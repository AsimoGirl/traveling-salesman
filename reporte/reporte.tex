\documentclass{article}

\usepackage{arxiv}
\usepackage[utf8]{inputenc} % allow utf-8 input
\usepackage{float}
\usepackage{listings}
\usepackage[T1]{fontenc}    % use 8-bit T1 fonts
\usepackage{hyperref}       % hyperlinks
\usepackage{url}            % simple URL typesetting
\usepackage{mathtools}
\usepackage{amssymb,mathrsfs}
\usepackage{booktabs}       % professional-quality tables
\usepackage{amsfonts}       % blackboard math symbols
\usepackage{dsfont}
\usepackage{nicefrac}       % compact symbols for 1/2, etc.
\usepackage{microtype}      % microtypography
\usepackage{lipsum}		% Can be removed after putting your text content
\usepackage[spanish]{babel}


\title{El Problema del Agente Viajero usando la heurística de Aceptación por
Umbrales \emph{(Threshold Accepting)}}

%\date{September 9, 1985}	% Here you can change the date presented in the paper title
%\date{} 					% Or removing it

\author{
  Sandra del Mar Soto Corderi\\
  No. cuenta: 315707267\\
}

% Uncomment to remove the date
%\date{}

% Uncomment to override  the `A preprint' in the header
%\renewcommand{\headeright}{Technical Report}
%\renewcommand{\undertitle}{Technical Report}

\begin{document}
\maketitle

\begin{abstract}
El Problema del Agente Viajero es un problema \emph{NP-Hard} que responde la siguiente pregunta;
``dada una lista de ciudades y las distancias entre cada par de ellas, ¿cuál es la ruta más corta
posible que visita cada ciudad exactamente una vez y al finalizar regresa a la ciudad origen?''

Que en su versión de optimización dice ``Dada una gráfica completa con pesos $G=(V,E)$, encontrar un ciclo
Hamiltonianto en $G$ con un peso mínimo (si acaso existe)''.

El método de Aceptación por Umbrales \emph{(Threshold Accepting)} es mucho más simple que Recocido
Simulado \emph{(Simmulated Annealing)} y es el motivo por el cual se usará para TSP.
\end{abstract}


\section{Introducción}
Para la resolución de TSP, primero se contruyó una gráfica $G=(V,E)$ con las ciudades y sus respectivas
distancias entre las ciudades con una base de datos que se nos fue dada. Esta gráfica ponderada
tiene una función de de peso para las aristas $w: E \to \mathds{R}^+$. Sea $S \subset V$ una
instancia de TSP que se quiera resolver.

Para hacer más sencillo la versión de optimización de este problema se hara uso de una
gráfica completa $G_s =(V_s, E_s)$, donde $V_s = S$ y $E_s= \{ (u,v) | u,v \in S \land u \not= v \}$,
con la función de peso amentada $w_s : E_s \to \mathds{R}^+$ definida como:
\[
  w_s(u,v) =
    \begin{cases}
      w(u,v)                   & \text{si }(u,v) \in E\\
      d(u,v) \times \max_d (S) &\text{e.o.c.}\\
    \end{cases}
\]

donde $d(u,v)$ es la distancia natural entre dos vértices $u$ y $v$; y $max_d (S)$ será la
distancia máxima de $S$.

La \textbf{distancia natural} se utiliza para calcular la distancia natural entre dos elementos
de $S$ para de esta manera tener nuestra gráfica completa, se utilizarán las coordenadas de las
ciudades correspondientes. Su fórmula está definida por
\[
    d(u,v) = R \times C
\]
donde $R$ es el radio de la Tierra y $C$ está definida como $C = 2 \times \arctan(\sqrt{A},\sqrt{1-A})$.

La \textbf{distancia máxima} de $S$ se define como:
\[
    \max_d (S) = \max \{ w(u,v) | u,s \in S \land (u,v) \in E \} 
\]

Una solución de TSP dada una instancia $S$, es cualquier permutación de de los elementos de $S$ y se
puede asegurar que estas soluciones son válidas para $G_s$. Dándonos así, dos tipos de permutaciones:
las factibles y las no factibles. Decimos que es factible si y sólo si las aristas entre dos elementos
consecutivos de la permutación existen en $E$. Si al menos una arista no existe, decimos que no es
factible.


La optimización de TSP necesitará una \textbf{función de costo} que evaluará ``qué tan buena (o no) es una
solución'' para así poder comparar burdamente las evaluación de ésta y así poder decidir cuál es
mejor utilizando el criterio de entre más \textit{pequeña} sea la evaluación es \textit{mejor}.
Para esto se va a normallizar la función de costo, pero para esto se necesitará un normalizador.

\textbf{El normalizador}, como su nombre lo dice, normalizar la función de costo y nos dirá si una
solución $S$ es factible si esta evaluada entre 0 y 1, y que una solución no factible se evalue con
un valor mayor que 1.

\textbf{Función de costo}, sea $S \subset V$ una instancia de TSP: la función de costo $f$ de una
permutación $P = v_{\rho(1)}, \ldots, v_{\rho(k)}$ de los elementos de $S$ está definida como:
\[
    f(P) = \frac{\sum_{i=2}^{k} w_s(v_{\rho(i-1)}, v_{\rho(i)})}{\mathcal{N}(S)}
\]

\section{Aceptación por Umbrales \emph{(Threshold Accepting)}}
Dado un problema $\mathscr{P}$ de optimización y clasificado como NP-duro, sea $S$ el conjunto de
posibles soluciones a una instancia de $\mathscr{P}$. Se supondrá que se tiene una función
$f: S \to \mathds{R}^+$, (función objetivo), tal que $0 \leq f(s) \leq \infty$ para cualquier
$s |in S$. Dadas $s,s'$ si $f(s) < f(s')$, entonces se considerará a la solición $s$ mejor que $s'$.

La idea central de \textbf{la aceptación por umbrales}, es dada una temperatura inicial
$T \in \mathds{R}^+$ y una solución inicial (obtenida de alguna manera). de forma aleatoria buscar
una solución vecina $s'$ tal que $f(s') \leq f(s) + T$, y entonces actualizar $s$ para que sea $s'$;
en este caso diremos que la solución $s'$ es \emph{aceptada}. Se continúa de esta manera mientras la
temperatura $T$ es disminuida paulatinamente siguiendo una serie de condiciones: el proceso termina
cuando $T < \epsilon$; cuando se han generado un determinado número de soluciones aceptadas; o
cuando otra serie de condiciones es satisfecha.


\section{Elaboración del programa}

Este proyecto se inició en el lenguaje de programación Rust

\section{Configuración del sistema}



\section{Resultados}

\subsection{Instancia con 40 ciudades}
Con los identificadores de las ciudades:
\begin{verbatim}
1,2,3,4,5,6,7,75,163,164,165,168,172,327,329,331,332,333,489,490,491,492,493,496,652,653,654,
656,657,792,815,816,817,820,978,979,980,981,982,984  
\end{verbatim}

Usando la semilla \textbf{635} nos da los siguientes resultados:
\begin{verbatim}
Ruta: [980, 327, 871, 331, 164, 984, 491, 492, 489, 4, 817, 978, 5, 6, 165, 3, 333, 981, 820,
332, 982, 816, 823, 7, 654, 490, 653, 656, 2, 661, 657, 168, 1, 815, 496, 172, 163, 329, 493, 
979]
Costo: 0.21751778855705842
¿Es factible?: true
\end{verbatim}

\subsection{Instancia con 150 ciudades}
Con los identificadores de las ciudades:
\begin{verbatim}
1,2,3,4,5,6,7,8,9,11,12,14,16,17,19,20,22,23,25,26,27,74,75,151,163,164,165,166,167,168,169,171,
172,173,174,176,179,181,182,183,184,185,186,187,297,326,327,328,329,330,331,332,333,334,336,339,
340,343,344,345,346,347,349,350,351,352,353,444,483,489,490,491,492,493,494,495,496,499,500,501,
502,504,505,507,508,509,510,511,512,520,652,653,654,655,656,657,658,660,661,662,663,665,666,667,
668,670,671,673,674,675,676,678,815,816,817,818,819,820,821,822,823,825,826,828,829,832,837,839,
840,871,978,979,980,981,982,984,985,986,988,990,991,995,999,1001,1003,1004,1037,1038,1073,1075
\end{verbatim}

Usando la semilla \textbf{0} nos da los siguientes resultados:
\begin{verbatim}
Ruta: [350, 840, 151, 828, 12, 1038, 340, 502, 339, 675, 186, 520, 16, 671, 179, 183, 346, 17,
164, 11, 501, 499, 347, 817, 23, 176, 668, 352, 978, 6, 5, 1004, 22, 185, 991, 333, 27, 353, 990,
171, 652, 1075, 483, 512, 821, 75, 444, 826, 511, 504, 825, 500, 985, 674, 999, 8, 662, 331, 837,
979, 493, 509, 329, 839, 2, 656, 667, 184, 507, 7, 823, 678, 816, 187, 982, 14, 181, 332, 345,
820, 26, 654, 653, 344, 490, 676, 665, 673, 173, 815, 1, 829, 832, 661, 663, 657, 508, 986, 9,
168, 505, 19, 496, 182, 172, 163, 351, 981, 3, 165, 988, 174, 4, 489, 25, 492, 491, 984, 995, 349,
1003, 334, 871, 343, 510, 660, 20, 327, 670, 336, 980, 297, 1001, 74, 822, 166, 658, 666, 818, 655,
819, 1073, 1037, 494, 495, 167, 328, 326, 169, 330]
Costo: 0.1668609785897421
¿Es Factible?: true
\end{verbatim}


\section{Comentarios}





\end{document}